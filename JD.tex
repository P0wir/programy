\documentclass[10pt,a4paper]{article}
\usepackage[MeX]{polski}
\usepackage[utf8]{inputenc}
\usepackage{graphicx}
\usepackage{amsmath} %pakiet matematyczny
\usepackage{amssymb} %pakiet dodatkowych symboli
\usepackage{hyperref}
\title{Kolokwium \LaTeX{} - rząd P107}
\author{Jan Nowak}
\date{26.11.2021}

\begin{document}
\maketitle
\tableofcontents

\section*{Wprowadzenie}

To jest wprowadzenie objaśniające nieco kolokwium. Zwróć uwagę, że jesteśmy w sekcji nienumerowanej. W sekcji tytułowej wpisz swoje prawdziwe imię i nazwisko. Jako datę wpisz ręcznie datę pisania kolokwium. W tytule znajduje się specjalny symbol. Ustaw czcionkę na 10 pkt, papier jako A4, dołącz pakiet polski. Spis treści jest na początku. Wszystkie odsyłacze powinny być klikalne.

\section{Tabele}

W tej sekcji numerowanej testujemy tabele. Wykonaj poniższą tabelkę. Jest ona w bieżącym miejscu tekstu. Pierwsza kolumna jest do lewej,druga do  środka. Sama tabela jest wyśrodkowana.

To jest odwołanie do tabeli \ref{tab:chev}.

\begin{table}[h]
    \centering
    \begin{tabular}{|l|c|} \hline
Segment &D \\\hline
Typy nadwozia &4-drzwiowy sedan\\\hline
Naped& przedni\\\hline
Długosc &4770 mm\\\hline
Szerokosc &1815 mm\\\hline
Skrzynia biegów &5-biegowa manualna\\\hline
Wysokosc& 1440 mm\\\hline
    \end{tabular}
    \caption{Chevrolet Epica}
    \label{tab:chev}
\end{table}

\section{Wzory matematyczne}

Tutaj testujemy wzory matematyczne. Wszystkie są wyśrodkowane w oddzielnym wierszu. Zwróć uwagę, że niektóre posiadają numery.

$$\prod_{i=1}^n i =n!$$
$$\left\{
1,\frac{1}{2}, \frac{1}{3}, \ldots, 
\frac{1}{n+1}
\right\}
$$

\begin{equation}
    \hat{T} = \widehat{T}, 
\end{equation}

$$\int _a^x \int _a^s f(y) \, dy \, ds$$

\end{document}